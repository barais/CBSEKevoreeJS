\section{Related Work}

Several component model exist for building dynamically adaptable web application. OSGi~\cite{hall2011osgi} provides an initial RFP for providing an OSGi for web applications. Eclipse provides an initial solutions within the Orion project to write plugins for its online IDE~\footnote{http://wiki.eclipse.org/Orion}. However, this approach does not support dynamic reconfiguration without reloading the web pages.   ComponentJS~\footnote{http://componentjs.com/} is a stand-alone MPL-licensed Open Source library for JavaScript, providing a powerful run-time Component System for hierarchically structuring the User-Interface (UI) dialogs of complex SPA. It provides a rich component model for UI composition based on the concepts  of Event, Service, Hook, Model, Socket and Property. However, it does not manage the dynamic reconfiguration of running applications. In~\cite{150010}, Lerner et al present C3, an implementation of the HTML/CSS/JS platform designed for web-client research and experimentation. C3 proposes explores the role of extensibility throughout the web platform for customization and research efforts. C3 proposes an interesting component model for the Web browser, (the application container) itself. It proposes an extensible architecture to let developer evolving the browser.  In~\cite{escoffier:hal-00854339}, escoffier et al developped a service-oriented component framework, named h-ubu. Its purpose is to bring modularity to applications and to ease their runtime adaptation. H-ubu is based on the notion of components with provided and required services, and on a hub, a specific component in charge with runtime components bindings. H-ubu really follows a dynamic services approaches. Main adaptations consists in reacting when a component becomes unavailable but the framework does not provide specific mechanism to automatically deploy and remove required components. The configuration model is not explicit, as in a service oriented architecture, each hub manages dynamically the bindings between component services.