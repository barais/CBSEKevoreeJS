\section{Main challenges for the use of the browser as a container for reconfigurable single page web applications }
A single-page application (SPA) is a web application that fits on a single web page, providing a more fluent user experience similar to a desktop application. 
This architecture style is now a standard way of designing modern Web applications, but still lacks support for dynamic reconfigurability. 
\ff{SPA has impact on browser caching, and enforce more easilly a dynamic web page construction based on data. Maybe we should extends a bit here to make this clear}
This section discusses eight important challenges to  enable dynamic reconfigurability in a SPA.   

\subsection{Automatically provision component implementation and third-party libraries}

Component-based distributed systems are known to be hard to deploy for two main reasons: the complexity of their structure and the complexity of the deployment tasks. Most of the current tools are not able to properly address these challenges because the component dependency descriptions they rely on lack expressiveness. Web modules does not avoid this drawback. Indeed if most of them use tools to manage dependencies at design time such as node package manager (npm\footnote{https://www.npmjs.com/}) or bower\footnote{http://bower.io/}, this tools are  generally not used to enable the dynamic loading of third parties library at runtime. Besides, deploying component in the web is generally done when the user access the web page. In Web applications, updates are made by downloading new software artefacts when the user refresh the whole page. In a SPA, either all necessary code -- HTML, JavaScript, and CSS -- is retrieved with a single page load, usually in response to user actions. The page does not reload at any point in the process, nor does control transfer to another page, although the location hash can be used to provide the perception and navigability of separate logical pages in the application. SPA Frameworks provide solutions to  avoid waiting for pages to load and to support dynamic updates of page fragments. However, this mechanism never includes the integration of the update of existing libraries. The first challenge for building dynamically adaptable single page web applications is to manage the dynamic deployment of new components and their third-parties libraries without perturbing the running applications. It includes also a correct management of libraries dependencies based on standard tooling used by developers such as bower or npm. 

\ff{maybe an idea to extend here would be to speak about CDNJS, and similar approach where they try to provision SPA without only the use of server. This approach is pretty similar to this idea of CBSE driven provisioning. Either we put a remark here and wait for the contribution, either we just introduce CDNJS}

\subsection{Type System}
The second main challenge is to make web components contract-aware~\cite{beugnard1999making}. If the CBSE community agrees to trust a component, we must be able to determine how this component will behave. The Web domain is far from the development method used for  mission-critical applications. Web developers generally use JavaScript as a programming language without providing a clear interface for their components. A component model for Web Application must force the developer to declare component interface and propose at minimum a basic level of contract to support duck typing verification when assembling components~\cite{beugnard1999making}. These contracts can also be used to check that the component interface fully conforms to its implementation if developers used  programming language such as TypeScript~\cite{rastogi2015safe} or Dart~\cite{dhiman2012google} to implement its components.
\ff{this could sounds like an attack point, if WebDev don't care about type, why they will use a modularized systems? :-) We should just turn and reverse the sentence to make this obviouss}

\subsection{User Interface composition}  
The next challenge is to provide a way to compose the User Interface (UI). If HTML-based interface technologies enable end-users to easily use various remote Web applications, it is difficult for end-users to express complex compositions. Currently, the main Web actors provide framework to express complex compositions, such as react~\cite{fedosejev2015react} from Facebook that provides an adaptation algorithm to compute the minimum diff in the DOM. Google's polymer~\footnote{https://www.polymer-project.org/1.0/} is a similar approach with stronger emphasis on reusability. RiotJS~\footnote{http://riotjs.com/} also provides a lightweight framework for UI composition, however limited to static composition. 
None of them are able to create dynamically reconfigurable user interface composition for the deployed components.
\ff{The last sentence is an attack point :-), should be protected}

\subsection{Security handling}
Providing reconfigurations support from an external manager leads to several security issues. 
If the SPA can receive a new configuration, automatically download components and update the running ones, it becomes easy to disseminate malware to any running application. 
The model which is the blueprint of the deployed runtime system should embed mechanisms to ensure authentication and data integrity of the system. 
\ff{it is not that this is not new, the claim is more that remote CBSE will not introdcuce new problems and therefore can rely on existing already deployed protecting solutions, such as HTTPS}
This kind of issue is not new to the world of web development and existing principles can be applied such as https protocols, Role-based access control\dots

\subsection{Search engine optimization}
Because of the lack of JavaScript execution on crawlers of some popular Web search engines, SEO (Search engine optimization) has historically presented a problem for public facing websites wishing to adopt the SPA model. In fact modern Web search engine such as google support the fact that a SPA can dynamically generate new contents~\cite{googlesearch}. However, in the case of a dynamic component model, the crawler cannot knows in advance all the components that can be installed in a web application. Consequently, search engine that knows only the first configuration of the application cannot correctly index such a dynamically reconfigurable SPA. This require to integrate new metadata to define, for example, all the components that can be installed on a running web apps. In that case, a search engine can do the hypothesis of the closed world to know all the potential configurations of a Web applications (even if it can become combinatorial).  
\ff{Definitevely true and a very interesing point. However, maybe we should turn it a little bit more positive. Like saying what could be open taks to pave the way towards a tighter integration of crawler and dynamic SPA (if we have to call them)}

\subsection{Client/Server code partitioning}
A dynamic component model for SPA must enable dynamic client/server code partitioning. In that sense such framework must provide abstractions for templating definition and rendering, abstraction for streams queries, … to let a developer change dynamically if the component is executed within the browser or at the server side. As it exists lots of technical stack at the server side that uses different programming languages, the component model must provide abstractions to define the component behavior that can be generated to this different technical stacks. In some simple case, we could imagine to do the hypothesis that the server also used JavaScript. In that case, it exists initial solutions \footnote{http://isomorphic.net/ references a set of solutions for client/server partitioning} to let share the same code for the client and the server side.  

\subsection{Browser history}
With an SPA being, by definition, "a single page", the model breaks the browser's design for page history navigation using the Forward/Back buttons. The traditional solution for SPAs has been to change the browser URL's hash fragment identifier in accord with the current screen state. 
The HTML5 specification has introduced \textit{pushState} and \textit{replaceState} providing programmatic access to the actual URL and browser history. 
For a dynamic component model, the main challenge is to defined how to aggregate the states of the many deployed components. 
The second challenge is to specify if a user request to go back to a previous state, does it include to come back to a previous configuration? 

\ff{Two-fold goal here. First as it is written the history of actions (or naviguation) and the second which is history for caching of data. Here we could speak about the second a bit using JS embedded db}

\subsection{Analytics}

Analytics is a common concerns for web applications. Tools such as Google Analytics rely heavily upon entire new pages loading in the browser, initiated by a URL change. As SPAs do not work this way, and in particular if we dynamically load new components analytics package has no idea who is doing what on the web applications. The new HTML5 history API allows to add page load events to an SPA. As the component web developers is in charge of using correctly this API, the challenge is then to avoid missing reports and double entries   
\subsection{Speed of initial load (overhead @ runtime) }
The last challenge is to guarantee the flexibility offered by the use of a pluggable architecture does not introduce too much overhead when we start the SPA. Such an approach must guarantee to  avoid excessive downloading of unused features. 
