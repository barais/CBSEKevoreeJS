\section{Conclusion}
This paper highlights the motivations, challenges, and main requirements to build a dynamic component model for single page applications. It shows how a distributed system running on top of various browsers relying on heterogeneous hardware can be considered as a common service. This leads to the need of managing the configuration of such a service from a common and abstract view. This paper presents the requirements for such a system and KevoreeJS, an implementation of the Kevoree Component Model for the Browser. It evaluates KevoreeJS by building a dashboard for sensor-based applications that can be dynamically reconfigured. In particular, it supports dynamic client/server code partitioning and dynamic component installation without refreshing the web page.

We are currently extending this approach to improve security, to support analytic services and search engine optimizations. From a technical point of view, we are working on a pattern to support other SPA frameworks such as AngularJS 2 or React. In this direction, we are working on a development model that decreases the coupling between the  component implementation and the SPA application framework used to provide a clean MVC framework.  %We are also building a common taxonomy and a survey to compare all the component models that exist for SPA based on the criteria discussed in section 2.
