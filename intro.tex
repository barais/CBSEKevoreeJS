\section{Introduction}

Traditional desktop-based productivity software moves to the Cloud. 
Nowadays, the browser is essentially an application container that allows users to run a single page application to  access a variety of Cloud-based applications and services that can be IDE, Word processor, Music Collection Manager, etc. 
A large offer of generative framework now propose to create the skeleton of such modern web applications, such as cite JHipster\footnote{https://jhipster.github.io/}, Mean.js~\footnote{http://meanjs.org}, ionic\footnote{http://ionicframework.com/}, keystonejs\footnote{http://keystonejs.com/}... 
These stacks generally use frameworks for developing the client part following a family of MVC pattern such as AngularJS~\cite{green2013angularjs}, emberjs~\cite{cravens2014building}, backbone~\cite{osmani2013developing}, durandal~\cite{monteiro2014learning}, react~\cite{fedosejev2015react}.  

%\ff{same remarks than abstract i would use another exemple to motivate the use of CBSE techniques}
A lesson learned from classical application development framework used for building IDE, word processor, operating systems, music or video player is the success of the use of the pluggable or composite architecture pattern~\cite{115158,schmidt2013pattern} as a core architecture concept,  i.e. a Software Architecture that allows dynamically plugging functionality using Pluggable Modules to tune its applications to its project requirements. In this trend, the OSGi framework specification~\cite{hall2011osgi} has been widely adopted by the Eclipse community and forms the basis of the Eclipse Runtime since Eclipse 3.X. The OSGi framework is a module system and service platform for the Java programming language that implements a complete and dynamic component model. Components (coming in the form of bundles for deployment) can be remotely installed, started, stopped, updated, and uninstalled without requiring a reboot. Developers use the same techniques for building entreprise services buses like apache camel to let an architect  
\ff{extends with an example of back-end technology?}

The design of highly configurable web applications requires the support of such pluggable architecture based on Component Based Software Engineering within the Browser. Current framework such as AngularJS, ember, backbone, durandal or react focus on a clear architecture of the web applications following a single page application principle~\cite{monteiro2014learning} but does not provide a solution to dynamically reconfigure a running application. Due to the increasing complexity of Web Applications and based on the experience of other applications container, it is now required to support the evolution of a software artefact or the installation of a new software artefact without reloading the web page.  
\ff{speak here about the composition of UI with reusable block, i.e. components ?}

This paper mainly highlights the challenges that arise when supporting a pluggable architecture pattern that enables the dynamic reconfiguration of a single page application. It also presents KevoreeJS in details, our approach to provide such a platform, and discuss its current limitations. We validate this work by showing how KevoreeJS can help to dynamically change Client/Server code partitioning in a dashboard for sensor-based system. Through this use case, we motivate the need for a dynamic module system for the browser, similar to OSGi for the JVM.  

The remainder of this paper is the following. Section 2 presents the main challenges for designing a module system for the JavaScript programming language that implements a complete and dynamic component model. Section 3 shows an overview of KevoreeJS and illustrates the main concepts through a motivating example of a dashboard for sensor based systems. Section 4 and 5 discuss related work and present ongoing work. 